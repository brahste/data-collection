%----------------------------------------------------------------------------------------
%	PACKAGES AND OTHER DOCUMENT CONFIGURATIONS
%----------------------------------------------------------------------------------------

\documentclass[
	12pt, % Default font size, values between 10pt-12pt are allowed
	%letterpaper, % Uncomment for US letter paper size
]{fphw}

% Template-specific packages
\usepackage[utf8]{inputenc} % Required for inputting international characters
\usepackage[T1]{fontenc} % Output font encoding for international characters
\usepackage[utf8]{inputenc}
\usepackage[sc]{mathpazo} % Add osf to arguments for cool numbers
\usepackage[scaled=0.90]{helvet}
\usepackage[scaled=0.85]{beramono}

\usepackage{subcaption}
\usepackage{graphicx} % Required for including images
\usepackage{booktabs} % Required for better horizontal rules in tables
\usepackage{listings} % Required for insertion of code
\usepackage{enumerate} % To modify the enumerate environment
\usepackage{pgfplots}
\pgfplotsset{compat=1.13}
\usepackage{graphicx}
\usepackage{enumitem}
\usepackage{hyperref}

%----------------------------------------------------------------------------------------
%	DOCUMENT INFORMATION
%----------------------------------------------------------------------------------------

\title{2020 Field Testing Manual} % Assignment title

\subtitle{Rover Setup, Connection, and ROS Configuration}

\author{Braden Stefanuk} % Student name

\date{\today} % Due date

\institute{Areospace Robotics Laboratory} % Institute or school name

%----------------------------------------------------------------------------------------

\begin{document}

\maketitle % Output the assignment title, created automatically using the information in the custom commands above

%----------------------------------------------------------------------------------------
%    CONTENT
%----------------------------------------------------------------------------------------

\section{Introduction}

This document is designed to be a self-contained directive on rover operation for future researchers and students at the Aerospace Robotics Laboratory (ARL). Some content is reiterated from the document \textit{ConnectingControllingHuskyAndArgo} by J.S Fiset (see resources at \url{https://github.com/brahste/data_collection}). However, substantial changes to the field testing procedures have been implemented, warranting a new manual to explain the setup, both experimental and configurational.

%----------------------------------------------------------------------------------------

\section{Sensors \& Hardware INCOMPLETE} 

Three primary sensors are equipped on the rover:
\begin{enumerate}
\item a ZED stereo-camera (ZED cam)
\item a VectorNav Inertial Measurement Unit (IMU)
\item FDTI current sensors for the front left and right wheels
\end{enumerate}
To incorporate the ZED cam into the sensor suite a laptop with a CUDA capable GPU is required. For this purpose, a Lenovo P53 with a GTX 1080 Ti is used.

%----------------------------------------------------------------------------------------

\section{Network Setup}

To command the rover from the Main Computer, ensure that it is connected through a hub. For our experiments a Cisco Wireless Access Point/Hub was used.

%----------------------------------------------------------------------------------------

\section*{Task 2}

\begin{problem}
you can use a box as such
\end{problem}

\end{document}
